\documentclass[solutions]{worksheet}
\usepackage[english]{babel}
\usepackage{amsmath}

\coursename{Galois Theory}
\semester{Fall 2025}
\weeknumber{1}
\worksheetnumbertotal{3}
\worksheetnumber{1}

\begin{document}

\begin{exercise}

  This is a very simple exercise. It doesn't have any sub-question.
  Is $57$ prime?

\end{exercise}

\begin{solution}

  Yes.

\end{solution}

\begin{exercise}

  This exercise has multiple sub-questions.
  
  \begin{questions}

    \question Give an explicit algebraic formula for $\cos \frac {2\pi} {17}$.

    \question Find a polynomial of which $\cos \frac \pi {17}$ is a root.

  \end{questions}

  These identities allowed Gau{\ss} to draw a \emph{heptadecagon} (polygon with
  seventeen sides) from a ruler-and-compass construction.

  \begin{questions}

    \question Can all polygons be thus obtained?

  \end{questions}

\end{exercise}

\begin{solution}

  These questions are related to some of the oldest problems in Mathematics.
  The problem of building a heptadecagon solely using a ruler and a compass was
  open for centuries. The first solution is due to Gau{\ss}. It was published
  in 1796 in his famous \emph{Disquisitiones arithmeticae}.

  \begin{answers}

    \answer We have
      \begin{align*}
        \cos\frac{2\pi}{17} = &   \frac{1}{16}\left(\sqrt{17}-1+\sqrt{34-2\sqrt{17}}\right)\\
                              & + \frac{1}{8} \left(\sqrt{17+3\sqrt{17}
                                                          - \sqrt{34-2\sqrt{17}}
                                                          - 2\sqrt{34+2\sqrt{17}}} \right).\\
      \end{align*}

    \answer One of the roots of
      \begin{align*}
        & 32768 X^{16} - 131072 X^{14} + 212992 X^{12} - 180224 X^{10} \\
        & + 84480 X^8 - 21504 X^6 + 2688 X^4 - 128 X^2 + X + 1
      \end{align*}
      is $\cos \frac {\pi} {17}$.

    \answer Unfortunately, no, by Wantzel's theorem.

  \end{answers}

\end{solution}

\end{document}
